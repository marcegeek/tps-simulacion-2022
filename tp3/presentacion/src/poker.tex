\subsubsection{Prueba de Póker}
Esta prueba examina en forma individual los dígitos del número pseudoaleatorio generado. La forma como esta prueba se
realiza es tomando 5 dígitos a al vez y clasificándolos como : Par, dos pares, tercia, póker, full, quintilla, y todos
diferentes.

Las probabilidades para cada una de las manos del póker diferentes se muestran enseguida:

\begin{enumerate}
    \item Todos diferentes = 0.3024
    \item Un par = 0.504
    \item Dos pares = 0.108
    \item Tercia = 0.072
    \item Póker = 0.009
    \item Full = 0.0045
    \item Quintilla = 0.0001
\end{enumerate}

Con las probabilidades anteriores y con el número de números pseudoaleatorios generados, se puede calcular la frecuencia
esperada de cada posible resultado, la cual al compararse con la frecuencia observada, produce el estadístico:

$X^7_0 = \sum_{i=1}^{7} {(FO_i-FE_i)^2 \over FE_i}$

Si $X^2_0 < X^2_{\alpha,6}$, entonces los números pasan la prueba, y no podemos rechazar la hipótesis de que los números
fueron generados aleatoriamente e independientemente.

\paragraph{Implementación}
En este trabajo utilizamos 5 números de cada muestra y los 7 casos descriptos más arriba.

Usamos una significación del 5\%, y revisamos las cantidades obtenidas para acumular o ignorar las que no sean
suficientes.

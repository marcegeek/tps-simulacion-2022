Con las gráficas obtenidas y las pruebas realizadas hemos descubierto que tanto el generador congruencial lineal (parámetros GCC) y el generador Mersenne Twister (MT19937) de Python parecen ser generadores bastante buenos, mientras que el GCL con valores arbitrarios y el generador por el método de los cuadrados medios no resultan buenos generadores. En cuanto al método de los cuadrados medios, es interesante mencionar que, trabajando con 10 dígitos, las gráficas obtenidas pueden mostrar una aparente aleatoriedad, sin embargo es durante las pruebas estadísticas donde pueden notarse sus puntos débiles. Como conclusión final, es claro que es necesario realizar más pruebas, tal vez con distintos niveles de significancia, para poder extraer conclusiones más exactas sobre la aleatoriedad de los generadores.
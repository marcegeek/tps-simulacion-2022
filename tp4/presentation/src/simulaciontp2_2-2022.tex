\documentclass{article}

\usepackage[utf8]{inputenc}
\usepackage[T1]{fontenc}
\usepackage[spanish]{babel}
\usepackage{times}
\usepackage[spanish,es-tabla]{babel}

\usepackage{color}
\definecolor{gray97}{gray}{.97}
\definecolor{gray75}{gray}{.75}
\definecolor{gray45}{gray}{.45}

\usepackage{listings}
\lstset{ frame=Ltb,
  framerule=0pt,
  aboveskip=0.5cm,
  framextopmargin=3pt,
  framexbottommargin=3pt,
  framexleftmargin=0.4cm,
  framesep=0pt,
  rulesep=.4pt,
  backgroundcolor=\color{gray97},
  rulesepcolor=\color{black},
%
  stringstyle=\ttfamily,
  showstringspaces = false,
  basicstyle=\small\ttfamily,
  commentstyle=\color{gray45},
  keywordstyle=\bfseries,
%
  numbers=left,
  numbersep=15pt,
  numberstyle=\tiny,
  numberfirstline = false,
  breaklines=true,
}

% minimizar fragmentado de listados
\lstnewenvironment{listing}[1][]
{\lstset{1}\pagebreak[0]}{\pagebreak[0]}

\lstdefinestyle{consola}
{basicstyle=\scriptsize\bf\ttfamily,
  backgroundcolor=\color{gray75},
}

\lstdefinestyle{C}
{language=Python,
}
\setcounter{secnumdepth}{4}

\titleformat{\paragraph}
\usepackage{arxiv}

\usepackage[utf8]{inputenc} % allow utf-8 input
\usepackage[T1]{fontenc}    % use 8-bit T1 fonts
\usepackage{hyperref}       % hyperlinks
\usepackage{url}            % simple URL typesetting
\usepackage{booktabs}       % professional-quality tables
\usepackage{amsfonts}       % blackboard math symbols
\usepackage{nicefrac}       % compact symbols for 1/2, etc.
\usepackage{microtype}      % microtypography
\usepackage{lipsum}
\usepackage{graphicx}
\usepackage{import}
\usepackage{float}
\usepackage{amsmath}
\usepackage{subcaption}
\graphicspath{ {./images/} }

\title{TP 2.2 - GENERADORES DE NÚMEROS PSEUDOALEATORIOS DE DISTINTAS DISTRIBUCIONES DE PROBABILIDAD}


\author{
    Abud Santiago Elias \\
    Legajo 47015 \\
    \texttt{ sabudvicco@gmail.com} \\
    %% examples of more authors
    \And
    Buchhamer Ariel \\
    Legajo 46217\\
    \texttt{arielbuchhamer1@outlook.com} \\
    \And
    Castellano Marcelo \\
    Legajo 39028 \\
    \texttt{marce.geek22@gmail.com} \\
    \And
    Navarro Franco \\
    Legajo 46387 \\
    \texttt{franconavarro1889@gmail.com} \\
%% \AND
%% Coauthor \\
%% Affiliation \\
%% Address \\
%% \texttt{email} \\
%% \And
%% Coauthor \\
%% Affiliation \\
%% Address \\
%% \texttt{email} \\
%% \And
%% Coauthor \\
%% Affiliation \\
%% Address \\
%% \texttt{email} \\
}


\begin{document}
  \maketitle
  \begin{abstract}
    El siguiente documento tiene por objetivo realizar un estudio sobre distintas distribuciones de probabilidad con sus respectivas características. De cada una se desarrollarán sus conceptos, se utilizará la tecnología para mostrar su funcionamiento y finalmente testear la generación de valores de las mismas.\end{abstract}

    \keywords{
    Simulación \and Números Pseudoaleatorios \and Distribución \and Probabilidad \and Discretas \and Continuas
    }



% keywords can be removed
%\keywords{First keyword \and Second keyword \and More}

  \section{Introducción}
  \label{sec:introducción}
  Se busca poder observar la generación de los distintos tipos de distribuciones a través de la generación
    de números pseudoaleatorios. Los tipos de distribuciones a desarrollar y analizar son tanto continuas como discretas, por lo que usaremos generadores específicos de cada uno.
    Luego de realizar las simulaciones, se procederá a aplicar el test de bondad Chi-Cuadrado a algunos de ellos para conocer su resultado.


  \section{Números Pseudoaleatorios}
  \label{sec:headings}
  Un número pseudoaleatorio no es más que el valor de una variable aleatoria x que tiene una distribución de probabilidad uniforme definida en el intervalo (0, 1).

  Los números pseudoaleatorios constituyen una parte realmente importante en la simulación de procesos estocásticos y generalmente se usan para generar el comportamiento de variables aleatorias, tanto continuas como discretas. Debido a que no es posible generar números realmente aleatorios, los consideramos como pseudoaleatorios, generados por medio de algoritmos determinísticos que requieren parámetros de arranque.

  Podemos asegurar con altos niveles de confiabilidad que el conjunto de números que utilizaremos en una simulación se comportan de manera muy similar a un conjunto de números totalmente aleatorios; por ello es que se les denomina números pseudoaleatorios.
\newpage
  \section{Test de Bondad Ji Cuadrado}
  El estadístico ji-cuadrado (o chi cuadrado), que tiene distribución de probabilidad del mismo nombre, sirve para someter a prueba hipótesis referidas a distribuciones de frecuencias. En términos generales, esta prueba contrasta frecuencias observadas con las frecuencias esperadas de acuerdo con la hipótesis nula. En este artículo se describe el uso del estadístico ji-cuadrado para probar la asociación entre dos variables utilizando una situación hipotética y datos simulados. Luego se describe su uso para evaluar cuán buena puede resultar una distribución teórica, cuando pretende representar la distribución real de los datos de una muestra determinada. A esto se le llama evaluar la bondad de un ajuste. Probar la bondad de un ajuste es ver en qué medida se ajustan los datos observados a una distribución teórica o esperada. Para esto, se utiliza una segunda situación hipotética y datos simulados.

  En el presente documento utilizaremos estre test en algunas distribuciones de probabilidades, los cuales dividiremos en 10 intervalos, agrupando los valores y determinando las frecuencias observadas y esperadas.
  Como manera de verificación se realizará la sumatoria de todos los valores observados y la sumatoria de todos los valores esperados, ya que cada uno de estos valores debería de poseer un valor de 1000, debido a que se generarán 1000 números pseudoaleatorios.
  Para la visualizacion del valor de chi-cuadrado esperado utilizaremos la tabla con 9 grados de libertad y un 95 por ciento de confianza y la
  formula para obtener el valor de chi-cuadrado observado sera:

  \begin{equation}
    x^2 = \sum_{i = 1}^{c}\frac{(o_{i}-e_{i})^2}{e_{i}}
    \end{equation}
  Donde e es el valor esperado y o es el valor observado en cada intervalo.

  \section{Distribuciones de Probabilidad}

    En teoría de la probabilidad y estadística, la distribución de probabilidad de una variable
    aleatoria es una función que asigna a cada suceso definido sobre la variable aleatoria, la
    probabilidad de que dicho suceso ocurra. La distribución de probabilidad está definida
    sobre el conjunto de todos el rango de valores de la variable aleatoria.

    La distribución de probabilidad está completamente especificada por la función de distribución, cuyo valor en cada x
    real es la probabilidad de que la variable aleatoria sea menor o igual que x.
    Los tipos de variables existentes son:
    \item \textbf{Variable aleatoria:} Es aquella cuyo valor es el resultado de un evento aleatorio. Lo que quiere decir que son
    los resultados que se presentan al azar en cualquier evento o experimento.
    \item \textbf{Variable aleatoria discreta:} Es aquella que solo toma ciertos valores (frecuentemente enteros) y que resulta
  principalmente del conteo realizado.
    \item \textbf{Variable aleatoria continua:} Es aquella que resulta generalmente de la medición y puede tomar cualquier valor
    dentro de un intervalo dado

  A continuación se desarrollarán distintos generadores de valores de variables aleatorias a partir de las distribuciones probabilidad, las cuales están clasificadas según el tipo de variable.

  \subsection{Distribuciones Continuas de Probabilidad}
  Una distribución continua describe las probabilidades de los posibles valores de una variable aleatoria continua. Una variable aleatoria continua es una variable aleatoria con un conjunto de valores posibles (conocido como el rango) que es infinito y no se puede contar.

  \subsubsection{Distribución Uniforme}
  Se caracteriza por ser constante, en un intervalo entre a y b sin contener al cero. Surge del estudio de las características de los errores por redondeo al registrar un conjunto de medidas sujetas a cierto nivel de precisión.
  Sobresale respecto a las técnicas de simulación por su simplicidad y por el hecho de que se puede emplear para simular variables aleatorias a partir de case cualquier tipo de distribución de probabilidad.

  La función de densidad de probabilidad de la distribución uniforme continua es:
  \begin{equation}
    f{x} = \left\{\begin{array}{lcc}
                  \frac{1}{b-a} &   a<x<b \\
                  \\0, &no&esta&en&(a, b)
                  \end{array}
    \right.
  \end{equation}

  En esta ecuación  X es una variable aleatoria definida en (a, b)
  El valor esperado y la varianza de una variable aleatoria uniformemente distribuida, se puede representar por:
  \begin{equation}
    EX = \frac{b+a}{2}
  \end{equation}
  \begin{equation}
    VX = \frac{(b-a)^2}{12}
  \end{equation}

      \paragraph{Método de la transformación inversa\newline}
  Para simular una distribución uniforme sobre cierto intervalo conocido (a,b) deberemos, en primer lugar, obtener la
  transformación inversa para la distribución uniforme acumulativa:

  \begin{equation}
    F(x) = \int_{0}^{a}\frac{1}{b-a}dt = \frac{x-a}{b-a} \ \text{  } \ 0 \leq F(x) \leq 1.
  \end{equation}

  Lo cual podemos deducir como su función inversa a:
  \begin{equation}
  x = a +(b-a)r \ \text{  }\ 0 \leq r \leq 1.
  \end{equation}

  En seguida, generamos un conjunto de números aleatorios correspondientes al rango de las probabilidades acumulativas,
  es decir, los valores de variables aleatorias uniformes definidas sobre el rango 0 a 1. Cada número aleatorio r determina,
  de manera única, un valor de la variable x uniformemente distribuida.

  \paragraph{Código en Python}
  \begin{lstlisting}[style = Python]
    def uniforme(a, b, rep: int = 1):
      x = []
      for i in range(rep):
          x.append(a + (b - a) * (round(random(), 4)))
      return x
  \end{lstlisting}
  Se ingresa como parámetro a y b que son valores que expresan los límites de las distribuciones uniformes.


  \paragraph{Análisis\newline}

    En la gráfica se puede observar la figura formada a partir de la distribución uniforme obtenida de
    la generación de números pseudoaleatorios y del método de la transformación inversa.
    En el eje de las abscisas se encuentran los valores valores que rondan el intervalo (a,b) y en el
    eje de las ordenadas la frecuencia absoluta en ese intervalo.
    \begin{figure}[h]
      \centering
      \includegraphics[width=0.6\textwidth]{generated/Uniforme}
      \caption{Frecuencias absolutas de valores generados dentro del rango ingresado.}\label{fig:figure}
    \end{figure}
    \newpage



  \paragraph{Test\newline}


  Valor de X2 obtenido = 345.28000000000003

  Con 9 grados de libertad y con un 95 porciento de confianza se obtiene un valor de Chi cuadrado de 16.92.

  Como X2 es mayor a 16.92, la distribución no pasó el test.

  \begin{table*}
    \begin{center}
      \begin{tabular}{| r | l | c |} \hline
      Valor Observado & Valor Esperado &  Valor Ji-Cuadrado \\ \hline
      45 & 100 & 30.25 \\
      38 & 100 & 68.69 \\
      34 & 100 & 112.25 \\
      39 & 100 & 149.46 \\
      40 & 100 & 185.46 \\
      40 & 100 & 221.46 \\
      39 & 100 & 258.67 \\
      50 & 100 & 283.67 \\
      44 & 100 & 315.03000000000003 \\
      45 & 100 & 345.28000000000003 \\ \hline
      \end{tabular}
      \caption{Valores observados,esperados y chi-cuadradado de una distribución Uniforme}
    \end{center}
  \end{table*}



  \subsubsection{Distribución Exponencial}

  En caso de que la ocurrencia de un evento en un intervalo corto, y si la ocurrencia de tal evento es, estadísticamente independiente respecto a la ocurrencia de otros eventos, entonces el intervalo de tiempo entre ocurrencias de eventos de este tipo estará distribuido en forma exponencial.
  Se dice que una variable aleatoria X tiene una distribución exponencial, si se puede definir a su función de densidad como:

  \begin{equation}
    f(x) = ae^{-ax}\label{eq:equation}
  \end{equation}
  con a > 0 y x >= 0.
  La función de distribución acumulativa de X está dada por:
  \begin{equation}
    F(x) = \int_{0}^{a} ae^{-at}dt = 1-e^{-ax},
  \end{equation}
  y la media junto con la variancia de X se pueden expresar como
  \begin{equation}
    EX = \int_{0}^{\infty} xae^{-ax}dx = \frac{1}{a}
  \end{equation}
  \begin{equation}
    VX = \int_{0}^{\infty} (x-\frac{1}{a})^2 = \frac{1}{a^2} = (EX)^2
  \end{equation}

  \paragraph{Método de la transformación inversa\newline}

  Existen muchas maneras para lograr la generación de valores de variables aleatorias exponenciales. Puesto que la
  distribución acumulativa existe explícitamente (vista en la ecuación (8)),la técnica de la transformación inversa nos permite desarrollar métodos directos para dicha generación.
  Debido a la simetría que existe entre la distribución uniforme sigue que la intercambiabilidad de F(x) y 1 - F(x). Por lo tanto:
  \begin{equation}
    r = e^{-ax}
  \end{equation}
  y consecuentemente
  \begin{equation}
    x = -\frac{1}{a}log r  = - EXlog r
  \end{equation}
  Por consiguiente, para cada valor del número pseudoaleatorio r, se determina un único valor para x. Los valores de
  x toman tan sólo magnitudes no negativas, debido a que el log (r) es ≤ 0 para ser 0 ≤ r ≤ 1 y además se ajustan a la
  función de densidad exponencial con un valor esperado x. Es importante notar que pese a que esta técnica parece en
  principio muy simple, es preciso recordar que en una computadora digital el cálculo logarítmico natural involucra una
  expansión en serie de potencias para cada valor de la variable aleatoria un informe que se debe generar.
  \paragraph{Código en Python}
  \begin{lstlisting}[style = Python]
    def exponencial(ex, rep: int = 1):
      x = []
      for i in range(rep):
          x.append(-ex * math.log(round(random(), 4)))
      return x

  \end{lstlisting}
  El parámetro ex que ingresa es la media de la distribución exponencial y cuyo valor es generado por el generador Mersenne twister propio de Python.

  \paragraph{Análisis\newline}

  En esta gráfica podemos observar la figura formada a partir de la distribución exponencial obtenida de la generación de números pseudoaleatorios y del método de la transformación inversa.
  En el eje de las abscisas se encuentran los valores obtenidos de las simulaciones con un lambda de valor 1/5 ,con una media de 5 y en el eje
  de las ordenadas la frecuencia absoluta en intervalos de valor 1.

  \begin{figure}[h]
    \centering
    \includegraphics[width=0.6\textwidth]{generated/Exponencial}
    \caption{Frecuencias absolutas de valores generados.}
  \end{figure}
  \newpage
  \paragraph{Test\newline}

  Valor de X2 obtenido = 22.515057814796155

  Se poseen 10 intervalos, por lo tantos tendremos 9 grados de libertad.

  Con 9 grados de libertad y con un 95 porciento de confianza se obtiene un valor de Chi cuadrado de 16.92. Como X2 es mayor, no pasó el test.
  \begin{table*}
    \begin{center}
      \begin{tabular}{| r | l | c |}\hline
      Valor Observado & Valor Esperado &  Valor Ji-Cuadrado \\ \hline
      50 & 58.23546641575128 & 0.0858121416985893 \\
      60 & 54.84409686709124 & 0.5705194964456324 \\
      45 & 51.65022530588559 & 1.4267693260858056 \\
      67 & 48.64235034471853 & 8.354956358122323 \\
      48 & 45.80964038483559 & 8.459687048069659 \\
      32 & 43.14189461068685 & 11.337210546652889 \\
      32 & 40.62950625597428 & 13.170075011033358 \\
      26 & 38.26342800891591 & 17.10050347381562 \\
      50 & 36.035139432151084 & 22.512369051487404 \\
      584 & 582.7482523739898 & 22.515057814796155\\ \hline
      \end{tabular}
      \caption{Valores observados,esperados y chi-cuadradado de una distribución Exponencial}
    \end{center}
  \end{table*}








  \subsubsection{Distribucion Gamma}
  Si un determinado proceso consiste de k eventos sucesivos y si el total del tiempo transcurrido para dicho proceso se puede considerar igual a la suma de k valores independientes de la variable aleatoria con distribución
  exponencial, cada uno de los cuales tiene un parámetro definido a la distribución de esta suma coincidirá con una distribución gamma con parámetros a y k.
  La función gamma está descrita mediante la siguiente función de densidad:
  \begin{equation}
    f(x) = \frac{a^{k}x^{(k-1)}e^{-ax}}{(k-1)!}
  \end{equation}
  donde a > 0, k > 9 y x se considera no negativo.
  Respecto a la media y la variancia de esta distribución, sus corrspondientes expresiones están formuladas como sigue:
  \begin{equation}
    EX = \frac{k}{a}
  \end{equation}
  \begin{equation}
    VX = \frac{k}{a^{x}}
  \end{equation}
  \paragraph{Código en Python}
  \begin{lstlisting}[style = Python]
   def gamma(k, a, rep: int = 1):
      x = []
      for i in range(1, rep):
          tr = 1
          for j in range(1, k):
              tr *= random()
          x.append(-(math.log(tr)) / a)
      return x
  \end{lstlisting}
  Los parámetros que se ingresan son k (cantidad de eventos sucesivos) y a (lambda).

  \paragraph{Análisis \newline}
  En esta gráfica podemos observar las figuras que se forman a partir de la distribución gamma obtenida por
  la generación de números pseudoaleatorios y del método de la transformación inversa. En el eje de las abscisas se
  encuentran los valores obtenidos de las simulaciones a partir de un lambda de valor 5 y con un total de 20 eventos
  sucesivos y en el eje de las ordenadas la frecuencia absoluta en intervalos de valor 0.01.

  \begin{figure}[h]
    \centering
    \includegraphics[width=0.6\textwidth]{generated/Gamma}
    \caption{Frecuencias absolutas de valores generados.}
  \end{figure}
  \newpage


  \subsubsection{Distribución Normal}
  La distribución normal basa su utilidad en el teorema del límite central. Este postula que, la distribución de probabilidad de la sima de N valores
  de variable aleatoria x independientes pero idénticamente distribuidos, con medias respectivas u y variancias σ se aproximan asintóticamente a una distribución normal, a medida
  que N se hace muy grande.

  En consecuencia, el teorema del límite central permite el empleo de distribuciones normales para representar medidas globales operadas sobre los efectos de causas (errores)
  aditivas distribuidas en forma independiente sin importar la distribución de probabilidad a que obedezcan las mediciones de causas individuales.

  Si la variable aleatoria X tiene una función de densidad f(x) dada como:

  \begin{equation}
    f(x) = \frac{1}{σ_{x}\sqrt {2\pi}}e^{\frac{-(x-\mu)^2}{2σ^2}}
  \end{equation}

  El valor esperado y la variancia de la distribución normal están dados por:
  \begin{equation}
    EX = \mu_{x}
  \end{equation}
  \begin{equation}
    EX = \sigma_{x}^2
  \end{equation}

  \paragraph{Método de la transformación inversa\newline}
  El procedimiento para simular valores normales utilizando computadoras requiere el uso de la suma de K valores de
  variables aleatorias distribuidos uniformemente; esto es la suma de r1, r2... rk con cada ri definida en el intervalo 0 <ri
  <1. Aplicando la convención notacional de la forma matemática del teorema central del límite, encontramos que:
  \begin{equation}
    \theta = \frac{a+b}{2} = \frac{0+1}{2} = \frac{1}{2},
  \end{equation}
  \begin{equation}
    \sigma = \frac{b-a}{\sqrt {12}} = \frac{1}{\sqrt {12}},
  \end{equation}
  \begin{equation}
    z = \frac{\sum_{i=2}^{K}r_{i}-K/2}{\sqrt {K/12}}
  \end{equation}
  Pero por definición, z es un valor de variable aleatoria con distribución normal estándar.
  Por lo tanto:

  \begin{equation}
    x = \sigma_{x}(\frac{12}{K})^{1/2}(\sum_{i=1}^{K}r_{i}-\frac{K}{2}) + \mu_{x}
  \end{equation}
  Por lo tanto, con la ecuación anterior, podemos proporcionar una formulación muy simple para generar valores de
  variable aleatoria normalmente distribuidos. Para generar un solo valor x bastará con sumar K números aleatorios
  definidos en el intervalo de 0 a 1. Este procedimiento se puede repetir tantas veces como valores de variables aleatorias
  normalmente distribuidos se quieran

  \paragraph{Código en Python}
  \begin{lstlisting}[style = Python]
    def normal(ex, stdx, rep: int = 1):
      x = []
      for i in range(rep):
          sm = 0
          for j in range(12):
              sm += random()
          x.append(stdx * (sm - 6) + ex)
      return x
  \end{lstlisting}
  Se puede observar que se ingresa como parámetro la media (ex) y la desviación estándar que es stdx.

  \paragraph{Análisis\newline}
  En esta gráfica podemos observar la figura que se forman a partir de la distribución normal obtenida por
  la generación de números pseudoaleatorios y del método de la transformación inversa. En el eje de las abscisas se
  encuentran los valores obtenidos de las simulaciones a partir de una media de valor 2.35 y y una desviación de estándar
  de valor 30 y en el eje de las ordenadas la frecuencia absoluta de los diferentes intervalos del histograma.
  \begin{figure}[h]
    \centering
    \includegraphics[width=0.6\textwidth]{generated/Normal}
    \caption{Frecuencias absolutas de valores generados.}
  \end{figure}
  \newpage

  \paragraph{Test\newline}

  Valor de X2 obtenido = 9.304212174355968
  Se poseen 10 intervalos, por lo tantos tendremos 9 grados de libertad.
  Con 9 grados de libertad y con un 95 porciento de confianza se obtiene un valor de Chi cuadrado de 16.92
  El valor de chi cuadrado obtenido: 9.304212174355968 es menor que 16.92 ,por lo tanto Paso el test.
  \begin{table}[h]
    \begin{center}
      \begin{tabular}{| r | l | c |} \hline
      Valor Observado & Valor Esperado &  Valor Ji-Cuadrado \\ \hline
      5 & 2.2746694468500497 & 3.2652773501741432 \\
      14 & 14.110294410228015 & 3.266139476546129 \\
      47 & 56.34629805925615 & 4.816432759710728 \\
      147 & 144.9788279442314 & 4.844610229048467 \\
      244 & 240.5186147918511 & 4.895001517797601 \\
      248 & 257.37370714101644 & 5.236397639964036 \\
      198 & 177.6535039813194 & 7.566663528115143 \\
      73 & 79.077766308295 & 8.033789055589368 \\
      21 & 22.685503247816598 & 8.159019744429111 \\
      2 & 4.190697350629602 & 9.304212174355968 \\ \hline
      \end{tabular}
      \caption{Valores observados,esperados y chi-cuadradado de una distribución Normal}
    \end{center}
  \end{table}



  \subsection{Distribuciones Discretas de Probabilidad}
  Se encuentra definido un número muy significativo de distribuciones de probabilidad para variables aleatorias que solamente toman valores discretos, esto es, enteros no negativos. La distribución acumulativa de probabilidad para una variable aleatoria discreta X se define de manera muy similar a la de la ecuación
  \begin{equation}
    F(x) = P(X \leq x) = \sum_{x=0}^{x}f(x)
    \end{equation}
  Donde f(x) es la frecuencia o función de probabilidad de X definida por valores enteros que:
  \begin{equation}
    F(x) = P(X= x)
  \end{equation}
  Para x=0,1,2,...

  Las distribuciones discretas de probabilidad son muy útiles cuando se las emplea como modelos estocásticos para ciertos procesos de conteo, ya sea sobre muestras finitas o no finitas, donde la presencia o ausencia de un atributo dicotómico está gobernada por el azar.

  Las secciones siguientes contienen la descripción de técnicas para la generación de valores de variables estocásticas a partir de la mayoría de las distribuciones discretas de probabilidad más conocidas:

  \subsubsection{Distribución Pascal}
  Cuando los procesos de ensayos de Bernoulli se repiten hasta lograr que ocurran k éxitos (k > 1), la variable aleatoria que caracteriza al número de fallas tendrá una distribución binomial negativa.
  Por consiguiente, los valores de variables aleatorias con distribución binomial negativa coinciden esencialmente con la suma de k valores de variable aleatoria con distribución geométrica; en este caso, k es un
  número entero y la distribución recibe el nombre de distribución de Pascal.

  La función de distribución de probabilidad para una distribución binomial negativa está dada por:

  \begin{equation}
    f(x) = (\frac{k+x-1}{x})p^{k}q^{x} \ \text{   } \ x = 0,1,2,...,
  \end{equation}
  donde k es el número total de éxitos en una sucesión de k + x ensayos, con x el número de fallas que ocurren antes de obtener k éxitos.

  El valor esperado y la variancia de X se representa con:

  \begin{equation}
    EX = \frac{kq}{p}
  \end{equation}
  \begin{equation}
    VX = \frac{kq}{p^2}
  \end{equation}

  \paragraph{Código en Python}
  \begin{lstlisting}[style = Python]
    def pascal(k, q, rep: int = 1):
      nx = []
      for i in range(rep):
          tr = 1
          for j in range(k):
              tr *= random()
          x = math.log(tr) // math.log(q)
          nx.append(x)
      return nx
  \end{lstlisting}
  Los parámetros ingresados son k que hace referencia a la cantidad de repeticiones hasta el éxito y q es la probabilidad de no tener éxito.


  \paragraph{Análisis\newline}
  En esta gráfica podemos observar la figura que se forma a partir de la distribución Pascal obtenida por
  la generación de números pseudoaleatorios. En el eje de las abscisas se encuentran los valores obtenidos de las simulaciones
  a partir de repeticiones de k éxitos con valor igual a 5 y con una probabilidad de exito de 0.6.

  Además en el eje de las ordenadas se puede observar la frecuencia absoluta de los diferentes intervalos del histograma.
  \begin{figure}[h]
    \centering
    \includegraphics[width=0.6\textwidth]{generated/Pascal}
    \caption{Frecuencias absolutas de valores generados.}
  \end{figure}










  \subsubsection{Distribución Binomial}
  La distribución binomial proporciona la probabilidad de que un evento o acontecimiento tenga lugar x veces en un conjunto de n ensayos,
  donde la probabilidad de éxito está dada por p. La función de probabilidad para la distribución binomial se puede expresar de la manera siguiente:

  \begin{equation}
    f(x) = \binom{n}{x} p^{x}q^{n-x}
  \end{equation}
  donde x se toma como un entero definido en el intervalo finito 0,1,2,...n, y al que se le asocio el valor q = (1-p).

  El valor esperado y la variancia de la variable binomial X son
  \begin{equation}
    EX = np
  \end{equation}

  \begin{equation}
  VX = npq
  \end{equation}

  \paragraph{Código en Python}
  \begin{lstlisting}[style = Python]
   def binomial(n, p, rep: int = 1):
      x = []
      for i in range(rep):
          y = 0
          for j in range(1, n):
              if (random() - p) < 0:
                  y += 1
          x.append(y)
      return x
  \end{lstlisting}

  N representa la cantidad de ensayos independientes de Bernoulli y p representa la probabilidad de éxito.

  \paragraph{Análisis\newline}
  En esta gráfica podemos observar la figura formada a partir de la distribución Binomial obtenida por la generación de
  números pseudoaleatorios. En el eje de las abscisas se
  encuentran los valores obtenidos de las simulaciones a partir de "n.ensayos independientes de Bernoully para los cuales
  la probabilidad de éxito es de 0.4. En el eje de las ordenadas se encuentra la frecuencia absoluta de los diferentes intervalos del
  histograma.
  \begin{figure}[h]
    \centering
    \includegraphics[width=0.6\textwidth]{generated/Binomial}
    \caption{Frecuencias absolutas de valores generados.}
  \end{figure}

  \paragraph{Test\newline}

  Valor de X2 obtenido = 5.584025744557111

  Se poseen 10 intervalos, por lo tantos tendremos 9 grados de libertad.

  Con 9 grados de libertad y con un 95 porciento de confianza se obtiene un valor de Chi cuadrado de 16.92

  El valor de chi cuadrado obtenido: 5.584025744557111 es menor que 16.92 ,por lo tanto Paso el test.
  \begin{table}[h]
    \begin{center}
      \begin{tabular}{| r | l | c |} \hline
      Valor Observado & Valor Esperado &  Valor Ji-Cuadrado \\ \hline
      2 & 1.2183998806820666 & 0.5013942927964039 \\
      18 & 16.347797144268114 & 0.6683754617207881 \\
      99 & 98.40878879839349 & 0.6719272855890921 \\
      246 & 270.47446286358735 & 2.886551639102344 \\
      358 & 344.0787593417779 & 3.449797512910062 \\
      217 & 204.45808031184686 & 4.219147147926512 \\
      55 & 57.02444084702785 & 4.291017396581979 \\
      5 & 7.475187463117927 & 5.1106025919773 \\
      0 & 0.4601497332212823 & 5.570752325198582 \\
      0 & 0.013273419358528216 & 5.584025744557111 \\ \hline
      \end{tabular}
      \caption{Valores observados,esperados y chi-cuadradado de una distribución Binomial}
    \end{center}
  \end{table}





  \newpage
  \subsubsection{Distribución Hipergeométrica}
  En teoría de la probabilidad la distribución hipergeométrica es una distribución discreta relacionada con muestreos
  aleatorios y sin reemplazo. Suponga que se tiene una población de N elementos de los cuales, d pertenecen a la categoría
  A y N-d a la B. La distribución hipergeométrica mide la probabilidad de obtener x elementos de la categoría A en una
  muestra sin reemplazo de n elementos de la población original.
  La función de probabilidad de una variable aleatoria con distribución hipergeométrica puede deducirse a través de
  razonamientos combinatorios y es igual a:

  \begin{equation}
    f(x) = \frac{\binom{N_{p}}{x}\binom{N_{q}}{n-x}}{\binom{N}{n}}
    \end{equation}
  con 0 =< x =< Np, y 0 =< n-x =<Nq. Donde x,n y N son enteros. El valor esperado y la variancia se caracterizan como sigue:
  \begin{equation}
    EX = np
    \end{equation}

  \begin{equation}
    VX = npq(\frac{N-n}{N-1})
    \end{equation}
  La generación de valores hipergeométricos involucra, substancialmente, la simulación de experimentos de muestreo sin reemplazo.

  \paragraph{Código en Python}
  \begin{lstlisting}[style = Python]
   def hipergeometrica(tn, ns, p, rep: int = 1):
      x = []
      for i in range(rep):
          tn1 = tn
          ns1 = ns
          p1 = p
          y = 0
          for j in range(1, ns1):
              if (random() - p1) > 0:
                  s = 0
              else:
                  s = 1
                  y += 1
              p1 = (tn1 * p1 - s) / (tn1 - 1)
              tn1 -= 1
          x.append(y)
      return x
  \end{lstlisting}
  La población total es indicada con el parámetro tn, la muestra seleccionada con ns y un índice de proporción de 0.4.

  \paragraph{Análisis\newline}

  En esta gráfica podemos observar la figura formada a partir de la distribución Hipergeometrica obtenida a
  por la generación de números pseudoaleatorios. En el eje de las abscisas se encuentran los valores obtenidos de las
  simulaciones por medio de una población N (en nuestro caso N = 5000000), una muestra de dicha población de 500 y un
  índice de proporción de 0.4, mientras que en el eje de las  ordenadas se observa la frecuencia absoluta de los
  diferentes intervalos del histograma.
  \begin{figure}[h]
    \centering
    \includegraphics[width=0.6\textwidth]{generated/Hipergeométrica}
    \caption{Frecuencias absolutas de valores generados.}
  \end{figure}






  \subsubsection{Distribución Poisson}
  Si tomamos una serie de n ensayos independientes de Bernoulli, en cada uno de los cuales se tenga una probabilidad p muy pequeña relativa a la ocurrencia de un cierto evento,
  a medida que n tiende al infinito, la probabilidad de x ocurrencias está dada por la distribución de Poisson

  \begin{equation}
    f(x) = e^{-\delta}\frac{\delta^x}{x!}
    \end{equation}
  con x = 0,1,2,... y \delta > 0.

  Esto sucede siempre y cuando permitamos que p se aproxime a cero de manera que se satisfaga la relación \delta = np.


  \paragraph{Código en Python}
  \begin{lstlisting}[style = Python]
    def poisson(lamb, rep: int = 1):
      x = []
      for i in range(rep):
          cont = 0
          tr = 1
          b = 0
          while tr - b >= 0:
              b = math.exp(-lamb)
              tr *= random()
              if tr - b >= 0:
                  cont += 1
          x.append(cont)
      return x
  \end{lstlisting}
  El parámetro que se ingresa a parte de las tiradas es el del valor de lambda.

  \paragraph{Análisis\newline}
  En esta gráfica podemos observar la figura que se formada a partir de la distribución Poisson obtenida a través de
  la generación de números pseudoaleatorios. En el eje de las abscisas se  encuentran los valores obtenidos de las simulaciones
  por un lambda de valor 50 y en el eje de las ordenadas la frecuencia absoluta de los diferentes intervalos del histograma.
  \begin{figure}[h]
  \centering
  \includegraphics[width=0.6\textwidth]{generated/Poisson}
  \caption{Frecuencias absolutas de valores generados.}
  \end{figure}


  \paragraph{Test\newline}

    Valor de X2 obtenido = 4.981640986645496

    Se poseen 30 intervalos, por lo tantos tendremos 29 grados de libertad.

    Con 29 grados de libertad y con un 95 porciento de confianza se obtiene un valor de Chi cuadrado de 42,56

    El valor de chi cuadrado obtenido: 4.981640986645496 es menor que 16.92 ,por lo tanto Paso el test.
\begin{table}[h]
\begin{center}
\begin{tabular}{| r | l | c |} \hline
Valor Observado & Valor Esperado &  Valor Ji-Cuadrado \\ \hline
0 & 0.07112803794031959 & 0.07112803794031959 \\
3 & 2.6146757209846765 & 0.1279132059085921 \\
24 & 31.268611202195082 & 1.81755360813301 \\
140 & 145.84172123781602 & 2.0515449977759217 \\
305 & 301.3950691932913 & 2.0946629099433007 \\
311 & 303.27871616598674 & 2.291241901041262 \\
170 & 159.84882289088642 & 2.93589098157238 \\
41 & 46.80178059878204 & 3.6551084490650516 \\
6 & 7.992563228096383 & 4.151858753648817 \\
0 & 0.8297822329966786 & 4.981640986645496 \\ \hline
\end{tabular}
\caption{Valores observados,esperados y chi-cuadradado de una distribución Poisson}
\end{center}
\end{table}







\subsubsection{Distribución Empírica Discreta}
  Sea X una variable aleatoria discreta con P(X=bi) = pi.
  En consecuencia, resulta evidente que un método para generar X en una computadora es aquel que genera un valor de
  baile ahora aleatoria y sujeto a una instrucción el informe, el intervalo (0, 1) y un conjunto de valores X= bi siempre
  que se satisfaga:
  \begin{equation}
    p1 + ... + p_{i-1} < r \leq p_{1} + ... p_{i}.
    \end{equation}
  Pese a que se han desarrollado un buen número de técnicas de búsqueda basadas en este método, en su gran mayoría
  requieren un programa relativamente complejos que a su vez emplean un tiempo de computación excesivo.
  Uno de los procedimientos más rápidos para generar valores de variable aleatoria discreta es el desarrollado por G.
  Marsaglia, quien presupone la disponibilidad de una computadora decimal cuyos bloques o palabras de memoria pueden
  referirse mediante números. Esta última característica en realidad constituye una propiedad de la gran mayoría de las
  computadoras actuales. Conviene hacer notar que si bien este método es extremadamente rápido, también requiere
  por lo menos una memoria de 1000 palabras. Existe otro método desarrollado también por Marsaglia que en forma
  alternativa utiliza mucho menos capacidad de memoria aunque incrementa ligeramente el tiempo de computación.

  \paragraph{Código en Python}
  \begin{lstlisting}[style = Python]
    def empirica_discreta(rep: int = 1):
      x = []
      p = [0.273, 0.037, 0.195, 0.009, 0.124, 0.058, 0.062, 0.151, 0.047, 0.044]
      for i in range(rep):
          a = 0
          z = 1
          for j in p:
              a += j
              if random() <= a:
                  break
              else:
                  z += 1
          x.append(z)
      return x
  \end{lstlisting}
  No se ingresara ningún parámetro, definiendo nosotros una serie de probabilidades cuya sumatoria da como resultado el valor de uno.


  \paragraph{Análisis\newline}
  En esta gráfica podemos observar la figura que se forma a partir de la distribución Empírica discreta obtenida por
  la generación de números pseudoaleatorios. En el eje de las abscisas
  se encuentran los valores obtenidos de las simulaciones desde diez probabilidades (0.273,0.037,0.195,0.009,0.124,0.058,
  0.062,0.151,0.047,0.044) cuya suma da un total de 1, y en
  el eje de las ordenadas la frecuencia absoluta de los diferentes intervalos del histograma.
    \begin{figure}[h]
      \centering
      \includegraphics[width=0.6\textwidth]{generated/Empírica}
      \caption{Frecuencias absolutas de valores generados.}
    \end{figure}


  \paragraph{Test\newline}

\begin{table}[h]
\begin{center}
\begin{tabular}{| r | l | c |} \hline
Valor Observado & Valor Esperado &  Valor Ji-Cuadrado \\ \hline
    277& 273 & 0.05860805860805861 \\
224 & 37 & 945.1667161667162 \\
242 & 195 & 956.4949212949214 \\
135 & 9 & 2720.4949212949214 \\
77 & 124 & 2738.3094374239536 \\
30 & 58 & 2751.826678803264 \\
11 & 62 & 2793.7782917064897 \\
4 & 151 & 2936.8842519713903 \\
0 & 47 & 2983.8842519713903 \\
0 & 44 & 3027.8842519713903 \\ \hline
\end{tabular}
\caption{Valores observados,esperados y chi-cuadradado de una distribución Empírica}
\end{center}
\end{table}
Valor de X2 obtenido = 3027.8842519713903

Se poseen 10 intervalos, por lo tantos tendremos 9 grados de libertad.

Con 9 grados de libertad y con un 95 porciento de confianza se obtiene un valor de Chi cuadrado de 16.92

Como X2 es mayor a 16.92, no pasó el test.

\section{Conclusiones}


\section{Referencias}
  \label{sec:referencias}
    https://astridmll.wordpress.com/2016/09/13/numeros-pseudoaleatorios-y-sus-caracteristicas/
    https://www.medwave.cl/link.cgi/Medwave/Series/MBE04/5266?ver=sindiseno
    https://ansenuza.unc.edu.ar/comunidades/bitstream/handle/11086.1/1346/DistribC3%B3n%20de%20Probabilidades%20Discretas.pdf?sequence=1&isAllowed=y
\end{document}

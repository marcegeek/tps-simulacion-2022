\documentclass{article}
\usepackage[utf8]{inputenc} % allow utf-8 input
\usepackage[T1]{fontenc}    % use 8-bit T1 fonts
\usepackage[spanish]{babel} % idioma español
\usepackage{hyperref}       % hyperlinks
\usepackage{url}            % simple URL typesetting
\usepackage{booktabs}       % professional-quality tables
\usepackage{amsmath}        % useful for serious math
\usepackage{amsfonts}       % blackboard math symbols
\usepackage{nicefrac}       % compact symbols for 1/2, etc.
\usepackage{microtype}      % microtypography
\usepackage{kpfonts}        % use the same fonts for text and math
\usepackage{graphicx}
\graphicspath{ {./images/} }
\usepackage{float}          % para fijar figuras con \begin{figure}[H] ... \end{figure}
\usepackage{import}         % para facilitar la inclusión de los demas archivos
     % se puede hacer \input porque no contiene dependencias locales
\import{../../../latex/}{formato-extra} % con dependencias locales (eg. arxiv.sty)
\import{../../../latex/}{autores}

\title{Ejemplo de documento}

% paquetes locales adicionales
\usepackage{lipsum} % para mostrar texto de ejemplo

\begin{document}
  \maketitle
  \begin{abstract}
    Un ejemplo de documento que muestra como se puede unificar el formato e importar archivos en \LaTeX con el paquete '\emph{import}'.
  \end{abstract}
  \keywords{\LaTeX \and paquete \emph{import}}

  \section{Una sección}
    \import{./}{unasec}

  \section{Otra sección}
    \import{./}{unasec}

    \subsection{Subsección}
      \subsubsection{Subsubsección}
        \import{./}{unasec}

    \subsection{Aun más}
      \import{./}{unasec}

  % para la bibliografía
  \nocite{*} % citar todas las referencias al final sin especificarlas en el texto, si no con \cite{...} donde corresponda
  \bibliographystyle{unsrt}
  \bibliography{main}
\end{document}

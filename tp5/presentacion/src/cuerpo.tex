\begin{abstract}
  En este trabajo estudiaremos dos modelos de simulación de eventos discretos, el primero es el comportamiento
  de líneas de espera. El cual es de gran ayuda para predecir el comportamiento de dichas lineas en
  situaciones del mundo real, desde la entrada y salida de autos de un estacionamiento hasta la utilización
  de una red distribuida de servidores a lo largo del mundo que alojan una página web para miles de usuarios.
  El siguiente será un modelo de inventario que nos permitirá saber los costos correspondientes al mantenimiento
  o de compras necesarias y será de gran ayuda para un inventario en la vida real
\end{abstract}


\section{Introducción}\label{sec:introduccion}
En el siguiente trabajo simularemos mediante un programa desarrollado en lenguaje Python, una cola simple, en la que
solamente habrá llegada de clientes, atendidos por un único servidor luego de haber realizado una espera
determinada en la cola y posteriormente partirán. Adicionalmente se simulará una cola con una cantidad de servidores predeterminada.
El modelo de inventario contará con el mismo formato y consistirá en un proceso por mes en el que se realiza una orden de compra
a los proveedores, mientras se trata de alcanzar la demanda de los clientes a través de distintos indicadores de escacez y
existencia. Finalmente se realizará una evaluación y se calcularán las estadísticas correspondientes.
A su vez estudiaremos la eficacia de dichos programas al comparar ciertas medidas de rendimiento observadas con las
teóricas computadas y a su vez con las obtenidas mediante una calculadora web, y de otra simulación desarrollada
en el aplicativo Anylogic.


\section{Marco teórico}\label{sec:marco-teórico}

\subsection{Proceso Estocástico}\label{subsec:proceso-estocástico}
En la teoría de la probabilidad, un proceso estocástico es un concepto matemático que sirve para representar
magnitudes aleatorias que varían con el tiempo o para caracterizar una sucesión de variables aleatorias (estocásticas)
que evolucionan en función de otra variable, generalmente el tiempo.
Cada una de las variables aleatorias del proceso tiene su propia función de distribución de probabilidad y pueden o no
estar correlacionadas entre sí.

Cada variable o conjunto de variables sometidas a influencias o efectos aleatorios constituye un proceso estocástico.
Un proceso estocástico Xt puede entenderse como una familia uniparamétrica de
variables aleatorias indexadas mediante el tiempo t.
Los procesos estocásticos permiten tratar procesos dinámicos en los que hay cierta aleatoriedad.


\section{Modelo M/M/1}\label{sec:modelo-m/m/1}
Un sistema de espera M/M/1 es aquel que considera un servidor, con tiempos exponenciales de servicio y entre llegadas
de clientes. La implicancia que los tiempos de servicio se distribuyan exponencial es que existe una preponderancia de
tiempos de servicio menores al promedio combinados con algunos pocos tiempos extensos. Un ejemplo de ello es lo que
sucede en las cajas de los bancos donde la mayoría de las transacciones requieren poco tiempo de proceso por parte del
cajero, no obstante algunas transacciones más complejas consumen bastante tiempo. Por otra parte afirmar que los tiempos
entre llegadas se distribuyen exponencial implica una preponderancia de tiempos entre llegadas menores que el promedio
en combinación con algunos tiempos más extensos. Lo anterior tiene relación con la aleatoriedad del proceso de llegada
de clientes que permite establecer la Propiedad de Falta de Memoria o Amnesia de la Distribución Exponencial y
con los conceptos presentados en el artículo Qué son las Líneas de Espera (Teoría de Colas), donde queda en evidencia
que la formación de las colas o filas esta asociada a la variabilidad del sistema.

En este contexto consideremos la siguiente notación, donde valores usuales para A y B son M (distribución exponencial)
y G (distribución general).

En el estudio con colas simples, cuya notación de Kendalls son M/M/1, para la primera, donde se posee un sólo serivor,
las distribuciones de arribo y de servicio proceden de manera Markoviana y su cola es infinita, y en el caso de la
siguientes M/M/C dónde solamente se añadiran mas servidores.
Para la distribución de arribos, se tiene un parámetro λ, la tasa de arribos de clientes, de él obtenemos la media de
clientes entrantes por unidad de tiempo 1/λ.
Por otro lado, la tasa de tiempo de serivicios µ y la media de tiempos de servicios 1/ µ.
A partir de estos dos parámetros, obtenemos la tasa de utilización del servidor ρ = 1/sµ, la relación entre los clientes
arribando y los clientes partiendo, siendo s el número de servidores, en nuestro caso s = 1 (luego se permitira variarlo).
Como la llegada de clientes y su tiempo en ser servidos son variables aleatorias, se pueden considerar
nuevas variables aleatorias, que son combinaciones lineales de las anteriores, por ejemplo:

Considere un sistema de colas de un solo servidor para el cual el intervalo entre llegadas son A1, A2, . . . veces con
variables aleatorias independientes e idénticamente distribuidas.

A partir de una única ejecución de la simulación con retrasos de los clientes D1, D2, . . . , Dn, es obvio que el
estimador de d(n) es:

\begin{equation} \label{eq:equation}
\hat{d}(n) = \frac{\sum_{i = 1}^{n}D_{i}}{n}
\end{equation}
que es solo el promedio de los Di's que se observaron en la simulación.

Una de las medidas para nuestro modelo simple es el número promedio esperado de clientes en la cola (pero sin
ser servida), denotada por q(n), donde la n es necesaria en la notación para indicar que este promedio se
toma durante el período de tiempo necesario para observar los n retrasos que definen
nuestra regla de parada.
Este es un tipo diferente de "promedio" a comparación del retraso promedio en cola, porque se toma el tiempo
(continuo), en lugar de los clientes (siendo este discreto).
Por lo tanto, necesitamos definir qué significa este número promedio de tiempo de
clientes en cola.
Para hacer esto, sea Q(t) el número de clientes en cola en el tiempo t, para cualquier número real t>=0 ,
y sea T(n) el tiempo requerido para observar nuestros n retrasos en la cola.
Entonces para cualquier tiempo t entre 0 y T(n), Q(t) es un entero no negativo. Además, si dejamos que pi sea la
proporción esperada (que estará entre 0 y 1) del tiempo que Q(t) es igual a i, entonces una definición
razonable de q(n) sería:
 \begin{equation}
   \label{eq:equation2}
q(n) = \sum_{i=0}^{\infty}ip_{i}
 \end{equation}
Por lo tanto, q(n) es un promedio ponderado de los posibles valores de i para la longitud de la cola Q(t),
siendo las áreas la proporción esperada de tiempo que la cola pasa en cada uno de sus posibles longitudes.
Para estimar q(n) a partir de una simulación, simplemente reemplazamos los pi con estimaciones de ellos y
obtenemos:
\begin{equation}
  \label{eq:equation3}
  \hat{q}(n) = \sum_{i=0}^{\infty}i\hat{p}_{i}
\end{equation}
donde \hat{p}i es la proporción observada (en lugar de la esperada) del tiempo durante la simulación
donde que había i clientes en la cola.
Computacionalmente, sin embargo, es más fácil reescribir \hat{q}(n) usando algunas consideraciones geométricas.
Si dejamos que Ti sea el total de tiempo durante la simulación donde la cola tiene una longitud i, entonces
T(n) = T0 + T1 + T2 + ∙ ∙ ∙ y \hat{p}i = Ti/T(n), por lo que podemos reescribir la ecuación 1.1 como:

\begin{equation}
  \label{eq:equation4}
\hat{q}(n)=\frac{\sum_{i=0}^{\infty}iT_{i} }{T(n)}
\end{equation}

El numerador de la ecuación anterior, representa el área bajo la función Q(t):
\begin{equation}
  \label{eq:equation5}
  \sum_{i=0}^{k-1}iT_{i} = \int_{0}^{T(n)}Q(t)dt
\end{equation}
Reemplazándolo en la ecuación anterior, nos queda:

\begin{equation}
  \label{eq:equation6}
  \hat{q}(n)=\frac{\int_{0}^{T(n)}Q(t)dt }{T(n)}
\end{equation}
Esta integral puede ser calculada como la suma de rectángulos formados por la base tiempo de cierta cantidad
de clientes en cola por la altura dicha cantidad de clientes en cola.

La proporción esperada de tiempo del servidor en estado ocupado u(n), deviene de la probabilidad de que
el servidor no esté vacío, pN>0 = 1 − p0. Para calcular su estimador u(n), primero se define la "función
ocupada":

\begin{equation}\label{eq:equation7}
  B(t)\left\{ \begin{array}{lcc}
                  1 &   si  & Nt>0  \\
                  \\ 0 &  si  & Nt=0
  \end{array}
  \right
\end{equation}

Entonces u(n) es la porción del tiempo total en la que B(t) = 1.
Al igual que a la medida anterior, podemos considerarlo como el área bajo B(t), así,
\begin{equation}\label{eq:equation8}
\hat{u}(n)=\frac{\int_{0}^{T(n)}B(t)}{T(n)}
\end{equation}
En definitiva, u(n) es la sumatoria de áreas rectangulares, donde la base es el tiempo en el que el servidor
está en un estado específico, y la altura es dicho estado, 0 o 1.

\section{Modelo de Inventario}\label{sec:modelo-m/m/c}



\section{Análisis de resultados}
\section{Conclusiones}
\section{Referencias}
  \label{sec:references}
    https://es.wikipedia.org/wiki/Proceso_estoc%C3%A1stico#:~:text=En%20la%20teor%C3%ADa%20de%20la,otra%20variable%2C%20generalmente%20el%20tiempo.

    Arash Mahdavi, Simulation Modeling Consultant, The AnyLogic Company M/M/1/∞/∞ (single-server
    queues). En The Art of Process-Centric Modeling with AnyLogic, pages 81–103. 2020. Disponible en https://www.anylogic.com/resources/books/the-art-of-process-centric-modeling-with-anylogic/

    Averill M. Law, W. David. Kelton Simulation of a single-server queueing system. En Simulation Modeling Analysis,
    pages 13–74. 199

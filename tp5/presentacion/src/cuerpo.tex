\begin{abstract}
  En este trabajo estudiaremos dos modelos de simulación de eventos discretos, el primero es el comportamiento
  de líneas de espera. El cual es de gran ayuda para predecir el comportamiento de dichas lineas en
  situaciones del mundo real, desde la entrada y salida de autos de un estacionamiento hasta la utilización
  de una red distribuida de servidores a lo largo del mundo que alojan una página web para miles de usuarios.
  El siguiente será un modelo de inventario que nos permitirá saber los costos correspondientes al mantenimiento
  o de compras necesarias y será de gran ayuda para un inventario en la vida real
\end{abstract}


\section{Introducción}\label{sec:introduccion}
En el siguiente trabajo simularemos mediante un programa desarrollado en lenguaje Python, una cola simple, en la que
solamente habrá llegada de clientes, atendidos por un único servidor luego de haber realizado una espera
determinada en la cola y posteriormente partirán. Adicionalmente se simulará una cola con una cantidad de servidores predeterminada.
El modelo de inventario contará con el mismo formato y consistirá en un proceso por mes en el que se realiza una orden de compra
a los proveedores, mientras se trata de alcanzar la demanda de los clientes a través de distintos indicadores de escacez y
existencia. Finalmente se realizará una evaluación y se calcularán las estadísticas correspondientes.
A su vez estudiaremos la eficacia de dichos programas al comparar ciertas medidas de rendimiento observadas con las
teóricas computadas y a su vez con las obtenidas mediante una calculadora web, y de otra simulación desarrollada
en el aplicativo Anylogic.


\section{Marco teórico}\label{sec:marco-teórico}

\subsection{Proceso Estocástico}\label{subsec:proceso-estocástico}
En la teoría de la probabilidad, un proceso estocástico es un concepto matemático que sirve para representar
magnitudes aleatorias que varían con el tiempo o para caracterizar una sucesión de variables aleatorias (estocásticas)
que evolucionan en función de otra variable, generalmente el tiempo. Cada una de las variables aleatorias del proceso
tiene su propia función de distribución de probabilidad y pueden o no estar correlacionadas entre sí.

Cada variable o conjunto de variables sometidas a influencias o efectos aleatorios constituye un proceso estocástico.
Un proceso estocástico {\displaystyle X_{t}}{\displaystyle X_{t}} puede entenderse como una familia uniparamétrica de
variables aleatorias indexadas mediante el tiempo t. Los procesos estocásticos permiten tratar procesos dinámicos en los
que hay cierta aleatoriedad.

\section{Modelo M/M/1}\label{sec:modelo-m/m/1}
Un sistema de espera M/M/1 es aquel que considera un servidor, con tiempos exponenciales de servicio y entre llegadas
de clientes. La implicancia que los tiempos de servicio se distribuyan exponencial es que existe una preponderancia de
tiempos de servicio menores al promedio combinados con algunos pocos tiempos extensos. Un ejemplo de ello es lo que
sucede en las cajas de los bancos donde la mayoría de las transacciones requieren poco tiempo de proceso por parte del
cajero, no obstante algunas transacciones más complejas consumen bastante tiempo. Por otra parte afirmar que los tiempos
entre llegadas se distribuyen exponencial implica una preponderancia de tiempos entre llegadas menores que el promedio
en combinación con algunos tiempos más extensos. Lo anterior tiene relación con la aleatoriedad del proceso de llegada
de clientes que permite establecer la Propiedad de Falta de Memoria o Amnesia de la Distribución Exponencial y
con los conceptos presentados en el artículo Qué son las Líneas de Espera (Teoría de Colas), donde queda en evidencia
que la formación de las colas o filas esta asociada a la variabilidad del sistema.

En este contexto consideremos la siguiente notación, donde valores usuales para A y B son M (distribución exponencial)
y G (distribución general).



\section{Modelo M/M/C}\label{sec:modelo-m/m/c}



\section{Análisis de resultados}
\section{Conclusiones}
\section{Referencias}
  \label{sec:references}
    https://es.wikipedia.org/wiki/Proceso_estoc%C3%A1stico#:~:text=En%20la%20teor%C3%ADa%20de%20la,otra%20variable%2C%20generalmente%20el%20tiempo.


\subsection{Modelo de Inventario en Python}
En esta simulación se realizaron las corridas del sistema en base a eventos desarrollados por distintas acciones hechas en un inventario.
Para esto se tendrán en cuenta los siguientes parámetros con los que se contará para poder llegar a nuestros objetivos:
\begin{itemize}
    \item Nivel de Inventario: El nivel del inventario actualmente.
    \item Altas: el máximo de stock al que se puede llegar.
    \item Bajas: el mínimo de stock al que se puede llegar.
    \item Costo de reposición: costo de reposición de stock en el inventario.
    \item Costo de mantenimiento: precio por el cual cada item se mantiene en stock.
    \item Costo de reserva: costo de cada item al realizar una orden de compra.
    \item Costo de faltante: costo por la falta de stock.
    \item Media entredemanda.
    \item Distribución de demanda.
    \item Costo total ordenado
    \item Area de existencia.
    \item Area de escacez.
\end{itemize}

El funcionamiento del sistema desarrollado consta de eventos que surgen por meses donde la demanda cambia en base a una distribución de probabilidad.
Esto hace que luego de un tiempo determinado se realize la evaluación del inventario con sus existencias o faltantes para cubrir la demanda y se pueda actuar frente a esto.
En caso de que ante la demanda el modelo de inventario tenga una existencia menor a las Bajas, entonces se programa un pedido de arribos con una cantidad elegida para que luego de unos días lleguen y el inventario se reponga.
Caso contrario, si el nivel de inventario es mayor a las bajas, sigue funcionando de manera habitual.

Además el sistema calcula los costos mencionados anteriormente para que el usuario pueda ver cual es el estado de los costos del inventario.


Una vez realizados los experimentos se buscará analizar los siguientes resultados de los mismos:
\begin{itemize}
    \item Costo de Orden
    \item Costo de Mantenimiento
    \item Costo de Faltante
    \item Costo Total
\end{itemize}
\documentclass{article}


\usepackage{arxiv}

\usepackage[utf8]{inputenc} % allow utf-8 input
\usepackage[T1]{fontenc}    % use 8-bit T1 fonts
\usepackage{hyperref}       % hyperlinks
\usepackage{url}            % simple URL typesetting
\usepackage{booktabs}       % professional-quality tables
\usepackage{amsmath}                               % useful for serious math
\usepackage{amsfonts}       % blackboard math symbols
\usepackage{nicefrac}       % compact symbols for 1/2, etc.
\usepackage{microtype}      % microtypography
\usepackage{kpfonts}                               % use the same fonts for text and math
\usepackage{graphicx}
\graphicspath{ {./images/} }
\usepackage{float}          % para fijar figuras con H

\author{
 Abud Santiago Elias \\
  Legajo 47015 \\
  \texttt{ sabudvicco@gmail.com} \\
  %% examples of more authors
   \And
 Buchhamer Ariel \\
  Legajo 46217\\
  \texttt{arielbuchhamer1@outlook.com} \\
  \And
 Castellano Marcelo \\
  Legajo 39028 \\
  \texttt{marce.geek22@gmail.com} \\
  \And
   Dolan Guillermo Patricio \\
   Legajo 46101\\
  \texttt{guillermo230899@gmail.com} \\
  \And
  Navarro Franco \\
  Legajo 46387 \\
  \texttt{franconavarro1889@gmail.com} \\
  %% \AND
  %% Coauthor \\
  %% Affiliation \\
  %% Address \\
  %% \texttt{email} \\
  %% \And
  %% Coauthor \\
  %% Affiliation \\
  %% Address \\
  %% \texttt{email} \\
  %% \And
  %% Coauthor \\
  %% Affiliation \\
  %% Address \\
  %% \texttt{email} \\
}


\title{TP 1.1 - SIMULACIÓN DE UNA RULETA}

\begin{document}
\maketitle
\begin{abstract}
  En este proyecto de investigación realizamos un estudio sobre el comportamiento de una ruleta enfocándonos en los números que salen en cada ronda y realizando los cálculos necesarios para comprender su funcionamiento. El objeto del estudio era tratar de encontrar patrones en las rondas a través del estudio de distintas tiradas (100 rondas seguidas), las cuales nos servían para comparar entre ellas y conocer mejor la naturaleza de la ruleta.

  Al conocer los resultados pudimos encontrar que hay ciertas gráficas (como son la del promedio y el desvío estándar) que se mantienen o rondan un intervalo de números específicos que los podremos ver y analizar más detalladamente en la sección en cuestión.

  Esto no significa que se pueda determinar el número que puede salir en la próxima ronda, pero podría permitirnos calcular un cierto rango de resultados que podríamos tener en cuenta para una apuesta.
\end{abstract}

% keywords can be removed
%\keywords{First keyword \and Second keyword \and More}

\section{Introducción}
La ruleta es un juego de azar típico de los casinos, cuyo nombre viene del término francés roulette, que significa ``ruedita'' o ``rueda pequeña''. Su uso como elemento de juego de azar, aún en configuraciones distintas de la actual, no está documentado hasta bien entrada la Edad Media. Es de suponer que su referencia más antigua es la llamada Rueda de la Fortuna, de la que hay noticias a lo largo de toda la historia, prácticamente en todos los campos del saber humano.

La ``magia'' del movimiento de las ruedas tuvo que impactar a todas las generaciones. La aparente quietud del centro, el aumento de velocidad conforme nos alejamos de él, la posibilidad de que se detenga en un punto al azar; todo esto tuvo que influir en el desarrollo de distintos juegos que tienen la rueda como base.

Las ruedas, y por extensión las ruletas, siempre han tenido conexión con el mundo mágico y esotérico. Así, una de ellas forma parte del tarot, más precisamente de los que se conocen como arcanos mayores.

Según los indicios, la creación de una ruleta y sus normas de juego, muy similares a las que conocemos hoy en día, se debe a Blaise Pascal, matemático francés, quien ideó una ruleta con treinta y seis números (sin el cero), en la que se halla un extremado equilibrio en la posición en que está colocado cada número. La elección de 36 números da un alcance aún más vinculado a la magia (la suma de los primeros 36 números da el número mágico por excelencia: seiscientos sesenta y seis).

Esta ruleta podía usarse como entretenimiento en círculos de amistades. Sin embargo, a nivel de empresa que pone los medios y el personal para el entretenimiento de sus clientes, no era rentable, ya que estadísticamente todo lo que se apostaba se repartía en premios (probabilidad de 1/36 de acertar el número y ganar 36 veces lo apostado).

En 1842, los hermanos Blanc modificaron la ruleta añadiéndole un nuevo número, el 0, y la introdujeron inicialmente en el Casino de Montecarlo. Ésta es la ruleta que se conoce hoy en día, con una probabilidad de acertar de 1/37 y ganar 36 veces lo apostado, consiguiendo un margen para la casa del 2.7\% (1/37).

Más adelante, en algunas ruletas (sobre todo las que se usan en países anglosajones) se añadió un nuevo número (el doble cero), con lo cual el beneficio para el casino resultó ser doble (2/38 o 5.26\%).

\section{Descripción del trabajo}
\label{sec:headings}
En este trabajo realizamos la simulación del comportamiento de una ruleta tipo Montecarlo, los números están comprendidos del 0 al 36 (37 números en total) mediante un programa hecho en Python 3.8, el cual carga una lista con los números que fueron apareciendo en las tiradas de la ruleta, asumiendo una distribución de probabilidad uniforme discreta.

Para la aparición de números aleatorios se utilizó la función random.randint() de la librería NumPy, usando 0 y 37 como parámetros, ya que nos ofrece los números buscados con una distribución uniforme.

Con esos números obtenidos, procedemos luego a calcular el promedio, el desvío, la varianza y la frecuencia relativa de cada ronda en una tirada (100 rondas).

Finalmente las guardamos en listas que luego se utilizan para las gráficas con la librería “matplotlib”, con la cual podemos observar el histórico de los promedios, desvíos, varianza y frecuencia de cada número.

Además de esto también se encuentran las gráficas de los resultados globales de todas las tiradas para analizar los resultados totales de cada fórmula mencionada.

\section{Marco teórico}
\subsection{Distribución uniforme discreta}
Función de probabilidad:
Se llama función de probabilidad de una variable aleatoria discreta X a la aplicación que asocia a cada valor de xi de la variable su probabilidad pi.
\begin{equation}
p_X(x_i) = P(X = x_i) = \frac{1}{n}
\end{equation}

Esperanza matemática:
También llamada valor esperado, es la sumatoria de las probabilidades de que exista un suceso aleatorio, multiplicado por el valor del suceso aleatorio.
\begin{equation}
\operatorname{E}[X] = \frac{a+b}{2}
\end{equation}

Varianza:
Es una medida de dispersión que representa la variabilidad de una serie de datos respecto a su media.
\begin{equation}
\operatorname{Var}(X) = \sigma^{2} = \frac{(b-a+1)^{2}-1}{12}
\end{equation}

Desvío estándar:
Es una medida que ofrece información sobre la dispersión media de una variable. La desviación estándar es siempre mayor o igual que cero.
\begin{equation}
\sigma = \sqrt{\operatorname{Var}(X)} = \sqrt{\frac{(b-a+1)^{2}-1}{12}}
\end{equation}

\subsection{Fórmulas muestrales}
Frecuencia relativa:
\begin{equation}
f_{i} = f_{r}(x_{i}) = \frac {n_{i}}{N}
\end{equation}

Esperanza matemática:
\begin{equation}
\operatorname{E}[X] = \sum_{i=1}^{n}x_{i}\,f_{i}=x_{1}f_{1}+x_{2}f_{2}+\cdots +x_{n}f_{n} = \frac{\sum_{i=1}^{n}x_{i}\,n_{i}}{N}
\end{equation}

Varianza:
\begin{equation}
\operatorname{Var}(X) = \sigma^{2} = \operatorname{E}\left[(X - \mu)^{2}\right] = \operatorname{E}\left[(X - \operatorname{E}[X])^{2}\right]
\end{equation}

Desvío estándar:
\begin{equation}
\sigma = \sqrt{\operatorname{Var}(X)} = \sqrt{\operatorname{E}\left[(X - \operatorname{E}[X])^{2}\right]}
\end{equation}

\pagebreak  % comenzar siguiente sección en una página aparte
\section{Gráficas}
\begin{figure}[H]
  \centering
  \includegraphics[width=0.3\textwidth]{frec_rel.pdf}
  %\caption{Frecuencias relativas}
  \label{fig:frec_rel}
\end{figure}

\begin{figure}[H]
  \centering
  \includegraphics[width=0.3\textwidth]{promedio.pdf}
  %\caption{Promedios}
  \label{fig:promedio}
\end{figure}

\begin{figure}[H]
  \centering
  \includegraphics[width=0.3\textwidth]{varianza.pdf}
  %\caption{Varianzas}
  \label{fig:varianza}
\end{figure}

\begin{figure}[H]
  \centering
  \includegraphics[width=0.3\textwidth]{desvio.pdf}
  %\caption{Desvíos}
  \label{fig:desvio}
\end{figure}

\section{Conclusiones}
En los resultados de la simulación podemos observar que tras varias tiradas el promedio de los números aparecidos rondan entre 15 y 20.

Sin embargo, luego de aproximadamente 14 rondas de juego este promedio comenzará a oscilar cerca del número promedio de su caso obtenido analíticamente.

El desvío estándar de los números, al igual que el promedio comenzará con números altos y luego se estabilizará entre aproximadamente 10 y 12.

La varianza por tirada está comprendida entre 95 y 140.

Las aproximaciones de los valores simulados a los valores obtenidos analíticamente muestran cuan útil puede llegar a ser la simulación, y qué datos nos brinda la misma.

\bibliographystyle{unsrt}  
%\bibliography{references}  %%% Remove comment to use the external .bib file (using bibtex).
\end{document}
